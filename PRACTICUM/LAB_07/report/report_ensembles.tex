\documentclass[a4paper, 11pt]{article}

%%% Работа с русским языком
\usepackage{cmap}					% поиск в PDF
\usepackage{mathtext} 				% русские буквы в формулах
\usepackage[T2A]{fontenc}			% кодировка		% кодировка исходного текста
\usepackage[english,russian]{babel}	% локализация и переносы

%%% Дополнительная работа с математикой
\usepackage{amsfonts,amssymb,amsthm,mathtools} % AMS
\usepackage{amsmath}
\usepackage{icomma} % "Умная" запятая: $0,2$ --- число, $0, 2$ --- перечисление

\usepackage{indentfirst} % Красная строка в начале абзацев

\usepackage{setspace} % Междустрочный интервал
\singlespacing

\usepackage[left=20mm, top=10mm, right=10mm, bottom=25mm, nohead, footskip=7mm]{geometry} % поля документа

%% Номера формул
%\mathtoolsset{showonlyrefs=true} % Показывать номера только у тех формул, на которые есть \eqref{} в тексте.

%% Шрифты
\usepackage{euscript}	 % Шрифт Евклид
\usepackage{mathrsfs} % Красивый матшрифт

%% Перенос знаков в формулах (по Львовскому)
\newcommand*{\hm}[1]{#1\nobreak\discretionary{}
    {\hbox{$\mathsurround=0pt #1$}}{}}

%%% Работа с картинками
\usepackage{graphicx}  % Для вставки рисунков
\graphicspath{{images/}{images2/}}  % папки с картинками
\setlength\fboxsep{3pt} % Отступ рамки \fbox{} от рисунка
\setlength\fboxrule{1pt} % Толщина линий рамки \fbox{}
\usepackage{wrapfig} % Обтекание рисунков и таблиц текстом

%%% Работа с таблицами
\usepackage{array,tabularx,tabulary,booktabs} % Дополнительная работа с таблицами
\usepackage{longtable}  % Длинные таблицы
\usepackage{multirow} % Слияние строк в таблице


\usepackage[utf8]{inputenc}
\usepackage[russian]{babel}
\usepackage{amsmath,amsfonts,amssymb,amsthm,mathtools} %AMS

\usepackage{hyperref}  % Гиперссылки
\usepackage[usernames,dvipsnames,svgnames,table,rgb]{xcolor}
\usepackage{enumitem} %Для нумерации списков
\usepackage{multicol} % Несколько колонок
\usepackage{multirow} % Несколько строк
%\usepackage{caption} % отступы между названием и объектом
%\captionsetup[images]{skip=2ex}

\usepackage{dsfont}

\DeclareMathOperator*{\argmax}{arg\,max}

\hypersetup{
    unicode=true,
    pdftitle={Практическое задание №1}, % Заголовок
    pdfauthor={Кузьмин Никита, ММП 317},
    pdfcreator={Кузьмин Никита, ММП 317},
    colorlinks=false, % false - ссылки в рамках; true - цветные ссылки
    linkcolor=red,   % внутренние ссылки
    citecolor=green, % на библиографию
    filecolor=magenta, % на файлы
    urlcolor=blue % на URL
}

\usepackage{fancyhdr}% загрузим пакет
%\pagestyle{fancy}% применим колонтитул

\begin{document}
    \hfill Кузьмин Никита, ММП, 317.
    
    \begin{center} \Large Отчет по заданию №3 "\textbf{Композиции алгоритмов для решения задачи регрессии}".
        
         Случайный лес и градиентный бустинг.
    \end{center}
    \tableofcontents
    \newpage
    \section{Введение}
        В данном документе представлен отчет о проделанных экспериментах по практическому заданию №3, анализ результатов. 
        
        Краткое описание задания: 
        
        
        Необходимо реализовать алгоритмы \textbf{RandomForestMSE} и \textbf{GradientBoostingMSE}, провести указанные ниже эксперименты на датасете данных о продажах недвижимости, проанализировать результаты.
   \section{Эксперименты}
   
   В этом блоке приведены все обязательные эксперименты, которые изложены в формулировке задания.
   
   Стандартный дизайн эксперимента: 
   \begin{itemize}
       \item Из данных был удален признак <<\textbf{id}>>, так как он не несет полезной информации для модели. Так же признак <<\textbf{date}>> был закодирован с помощью \textbf{LabelEncoder} таким образом: ранним датам соответствуют меньшие значения. Датасет был разделен на тренировочную ($70\%$) и валидационную выборки ($30\%$).
       \item Стандартные параметры экспериментов в случае \textbf{RandomForestMSE} (если не оговорено обратное):
       \begin{itemize}
           \item n\_estimators = 20
           \item max\_depth = None (не ограничена)
           \item feature\_subsample\_size = 6
       \end{itemize}
       item Стандартные параметры экспериментов в случае \textbf{GradientBoostingMSE} (если не оговорено обратное):
       \begin{itemize}
           \item n\_estimators = 20
           \item max\_depth = 5
           \item feature\_subsample\_size = 6
           \item learning\_rate = 0.1
       \end{itemize}
   \end{itemize}
    
\end{document}