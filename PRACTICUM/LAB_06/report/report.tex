\documentclass[a4paper, 11pt]{article}

%%% Работа с русским языком
\usepackage{cmap}					% поиск в PDF
\usepackage{mathtext} 				% русские буквы в формулах
\usepackage[T2A]{fontenc}			% кодировка		% кодировка исходного текста
\usepackage[english,russian]{babel}	% локализация и переносы

%%% Дополнительная работа с математикой
\usepackage{amsfonts,amssymb,amsthm,mathtools} % AMS
\usepackage{amsmath}
\usepackage{icomma} % "Умная" запятая: $0,2$ --- число, $0, 2$ --- перечисление

\usepackage{indentfirst} % Красная строка в начале абзацев

\usepackage{setspace} % Междустрочный интервал
\singlespacing

\usepackage[left=20mm, top=10mm, right=10mm, bottom=25mm, nohead, footskip=7mm]{geometry} % поля документа

%% Номера формул
%\mathtoolsset{showonlyrefs=true} % Показывать номера только у тех формул, на которые есть \eqref{} в тексте.

%% Шрифты
\usepackage{euscript}	 % Шрифт Евклид
\usepackage{mathrsfs} % Красивый матшрифт

%% Перенос знаков в формулах (по Львовскому)
\newcommand*{\hm}[1]{#1\nobreak\discretionary{}
    {\hbox{$\mathsurround=0pt #1$}}{}}

%%% Работа с картинками
\usepackage{graphicx}  % Для вставки рисунков
\graphicspath{{images/}{images2/}}  % папки с картинками
\setlength\fboxsep{3pt} % Отступ рамки \fbox{} от рисунка
\setlength\fboxrule{1pt} % Толщина линий рамки \fbox{}
\usepackage{wrapfig} % Обтекание рисунков и таблиц текстом

%%% Работа с таблицами
\usepackage{array,tabularx,tabulary,booktabs} % Дополнительная работа с таблицами
\usepackage{longtable}  % Длинные таблицы
\usepackage{multirow} % Слияние строк в таблице


\usepackage[utf8]{inputenc}
\usepackage[russian]{babel}
\usepackage{amsmath,amsfonts,amssymb,amsthm,mathtools} %AMS

\usepackage{hyperref}  % Гиперссылки
\usepackage[usernames,dvipsnames,svgnames,table,rgb]{xcolor}
\usepackage{enumitem} %Для нумерации списков
\usepackage{multicol} % Несколько колонок
\usepackage{multirow} % Несколько строк
%\usepackage{caption} % отступы между названием и объектом
%\captionsetup[images]{skip=2ex}

\usepackage{dsfont}

\DeclareMathOperator*{\argmax}{arg\,max}

\hypersetup{
    unicode=true,
    pdftitle={cheat_sheet}, % Заголовок
    pdfauthor={Кузьмин Никита, ММП 317},
    pdfcreator={Кузьмин Никита, ММП 317},
    colorlinks=false, % false - ссылки в рамках; true - цветные ссылки
    linkcolor=red,   % внутренние ссылки
    citecolor=green, % на библиографию
    filecolor=magenta, % на файлы
    urlcolor=blue % на URL
}



%\usepackage{titlesec}

%\titleformat*{\section}{\LARGE\bfseries}
%\titleformat*{\subsection}{\Large\bfseries}
%\titleformat*{\subsubsection}{\large\bfseries}

\usepackage{sectsty}
\sectionfont{\LARGE}
\subsectionfont{\LARGE}
\subsubsectionfont{\Large}


\usepackage{fancyhdr}% загрузим пакет
%\pagestyle{fancy}% применим колонтитул

\begin{document}
    \hfill Кузьмин Никита, ММП, 317.
    
    \begin{center} \Large Отчет по практическому заданию №2 "\textbf{Применение линейных моделей для определения токсичности комментария}". 
    
    Логистическая регрессия и градиентный спуск.
    \tableofcontents
    \end{center}
    
    \newpage
    \section{Введение}
    
    В данном документе представлен отчет о проделанных экспериментах по практическому заданию №2, анализ результатов. 
    Краткое описание задания: необходимо реализовать линейный классификатор с произвольной функцией потерь.
    
    \section{Эксперименты}
    В этом блоке приведены все обязательные эксперименты, которые изложены в формулировке задания.
    Все эксперименты проводились на упрощенном датасете (рассматривается задача бинарной классификации) из соревнования \textbf{Toxic Comment Classification Challenge}, в котором нужно определить токсичность комментария. 
    
    Стандартный дизайн эксперимента: 
    \begin{itemize}
        \item Оценка качества и подбор параметров модели проводились на каждой эпохе с помощью отложенной тренировочной выборки (30\%). Все графики ниже построены по значениям accuracy, посчитанным на отложенной выборке.
        \item В тренировочную выборку был добавлен признак, состоящий из всех единиц, который позволяет учитывать смещение (\textbf{bias}). Было решено не использовать смещение в $L2$-регуляризации, чтобы даже при плохом выборе коэффициента регуляризации решающая гиперплоскость не вырождалась в 0.
        \item В стохастическом градиентном спуске проверяется критерий останова на каждой эпохе (не итерации).
    \end{itemize}

        \subsection{Исследование поведения градиентного спуска}
            Обновления весов модели при использовании градиетного спуска происходит по следующей формуле:
            \begin{equation}\label{exp1:weight_upd}
            w_t = w_{t-1} - \frac{\alpha}{t^\beta} \times \frac{1}{N} \times \sum_{i=1}^{N}\nabla_{w}\L(x_i, y_i|w_{t-1}),
            \end{equation}
            где $t$ - номер итерации, $\beta$ - \textbf{step\_beta}, $\nabla_{w}\L(x_i, y_i|w_{t-1})$ - градиент функции потерь.
            \subsubsection{Параметр размера шага \textbf{step\_alpha}}
                Параметр \textbf{step\_alpha $(\alpha)$} используется в градиентном спуске при обновлении весов в формуле \ref{exp1:weight_upd}.
                Рассмотрим следующие зависимости при разных значениях параметра \textbf{step\_alpha}:
                    \begin{enumerate}\label{exp:dependencies}
                        \item зависимость значения функции потерь от реального времени работы метода
                        \item зависимость значения функции потерь от итерации метода
                        \item зависимость точности (accuracy) от реального времени работы метода
                        \item зависимость точности (accuracy) от итерации метода
                     \end{enumerate}
                 Соответствующие графики приведены на
            \subsubsection{Параметр размера шага step\_beta}
                Параметр \textbf{step\_beta $(\beta)$} используется в градиентном спуске при обновлении весов в формуле \ref{exp1:weight_upd}.
                Аналогично предыдущему пункту рассмотрим зависимости из \ref{exp:dependencies} при разных значениях параметра \textbf{step\_beta} и проанализруем соответсвующие графики:
                
            
\end{document}