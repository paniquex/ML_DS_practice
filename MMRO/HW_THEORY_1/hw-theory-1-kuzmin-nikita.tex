\documentclass[a4paper, 11pt]{article}

%%% Работа с русским языком
\usepackage{cmap}					% поиск в PDF
\usepackage{mathtext} 				% русские буквы в формулах
\usepackage[T2A]{fontenc}			% кодировка		% кодировка исходного текста
\usepackage[english,russian]{babel}	% локализация и переносы

%%% Дополнительная работа с математикой
\usepackage{amsfonts,amssymb,amsthm,mathtools} % AMS
\usepackage{amsmath}
\usepackage{icomma} % "Умная" запятая: $0,2$ --- число, $0, 2$ --- перечисление

\usepackage{indentfirst} % Красная строка в начале абзацев

\usepackage{setspace} % Междустрочный интервал
\singlespacing

\usepackage[left=20mm, top=10mm, right=10mm, bottom=25mm, nohead, footskip=7mm]{geometry} % поля документа

%% Номера формул
%\mathtoolsset{showonlyrefs=true} % Показывать номера только у тех формул, на которые есть \eqref{} в тексте.

%% Шрифты
\usepackage{euscript}	 % Шрифт Евклид
\usepackage{mathrsfs} % Красивый матшрифт

%% Перенос знаков в формулах (по Львовскому)
\newcommand*{\hm}[1]{#1\nobreak\discretionary{}
    {\hbox{$\mathsurround=0pt #1$}}{}}

%%% Работа с картинками
\usepackage{graphicx}  % Для вставки рисунков
\graphicspath{{images/}{images2/}}  % папки с картинками
\setlength\fboxsep{3pt} % Отступ рамки \fbox{} от рисунка
\setlength\fboxrule{1pt} % Толщина линий рамки \fbox{}
\usepackage{wrapfig} % Обтекание рисунков и таблиц текстом

%%% Работа с таблицами
\usepackage{array,tabularx,tabulary,booktabs} % Дополнительная работа с таблицами
\usepackage{longtable}  % Длинные таблицы
\usepackage{multirow} % Слияние строк в таблице


\usepackage[utf8]{inputenc}
\usepackage[russian]{babel}
\usepackage{amsmath,amsfonts,amssymb,amsthm,mathtools} %AMS

\usepackage{hyperref}  % Гиперссылки
\usepackage[usernames,dvipsnames,svgnames,table,rgb]{xcolor}
\usepackage{enumitem} %Для нумерации списков
\usepackage{multicol} % Несколько колонок
\usepackage{multirow} % Несколько строк
%\usepackage{caption} % отступы между названием и объектом
%\captionsetup[images]{skip=2ex}

\usepackage{dsfont}

\DeclareMathOperator*{\argmax}{arg\,max}

\hypersetup{
    unicode=true,
    pdftitle={cheat_sheet}, % Заголовок
    pdfauthor={Кузьмин Никита, ММП 317},
    pdfcreator={Кузьмин Никита, ММП 317},
    colorlinks=false, % false - ссылки в рамках; true - цветные ссылки
    linkcolor=red,   % внутренние ссылки
    citecolor=green, % на библиографию
    filecolor=magenta, % на файлы
    urlcolor=blue % на URL
}

\usepackage{fancyhdr}% загрузим пакет
%\pagestyle{fancy}% применим колонтитул

\begin{document}
    \hfill Кузьмин Никита, 317, ММП.
    \begin{center}
        \large Теоретическое задание 1: матричные вычисления и матричное дифференцирование.
    \end{center}

    \noindent \large\textbf{Задание №1:} \normalsize
    
    Докажите тождество Вудбери:
    \[(A + UCV)^{-1} = A^{-1} - A^{-1}U(C^{-1} + VA^{-1}U)^{-1}VA^{-1},\]
    
    где $A \in \mathbb{R}^{n \times n}, \; C \in \mathbb{R}^{m \times m}, \; U \in \mathbb{R}^{n \times m}, \; V \in \mathbb{R}^{m \times n}, \; det(A) \ne 0, \; det(C) \ne 0$
    
    \textbf{Доказательство:}
    
    \quad Домножим правую часть равенства на $(A + UCV)$ и докажем, полученное выражение равно  $I$:
    \begin{equation}
        \begin{aligned}
            & (A + UCV)(A^{-1} - A^{-1}U(C^{-1} + VA^{-1}U)^{-1}VA^{-1}) = \\
            & = I + UCVA^{-1} - (U + UCVA^{-1}U)(C^{-1} + VA^{-1}U)^{-1}VA^{-1} = \\ 
            & = I + UCVA^{-1} - UC(C^{-1} + VA^{-1}U)(C^{-1} + VA^{-1}U)^{-1}VA^{-1} = \\
            & = I + UCVA^{-1} - UCVA^{-1} = I
        \end{aligned}
    \end{equation}
    
    \quad Что и требовалось доказать.
    \vspace{5ex}
    
    \noindent \large\textbf{Задание №2:} \normalsize
    
    
    Упростите каждое из следующих выражений:
    \begin{enumerate}
        \item $||uv^{T} - A||_{F}^2 - ||A||_{F}^2, где u \in u \in \mathbb{R}^m, v \in \mathbb{R}^n, A \in \mathbb{R}^{m \times n}$
    \end{enumerate}
    
    
    
    
    
\end{document}